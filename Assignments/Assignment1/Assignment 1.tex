\documentclass{article}

\usepackage{fancyhdr}
\usepackage{extramarks}
\usepackage{amsmath}
\usepackage{amsthm}
\usepackage{amsfonts}
\usepackage{tikz}
\usepackage[plain]{algorithm}
\usepackage{algpseudocode}
\usepackage{tikz,pgfplots,multicol,lstautogobble}

\usetikzlibrary{automata,positioning}

%
% Basic Document Settings
%

\topmargin=-0.45in
\evensidemargin=0in
\oddsidemargin=0in
\textwidth=6.5in
\textheight=9.0in
\headsep=0.25in

\linespread{1.1}

\pagestyle{fancy}
\lhead{\hmwkAuthorName}
\chead{\hmwkClass\ (\hmwkClassInstructor\ \hmwkClassTime)}
\rhead{\hmwkTitle}
\lfoot{\lastxmark}
\cfoot{\thepage}

\renewcommand\headrulewidth{0.4pt}
\renewcommand\footrulewidth{0.4pt}

\setlength\parindent{0pt}

\setcounter{secnumdepth}{0}
\newcounter{partCounter}
\newcounter{homeworkProblemCounter}
\setcounter{homeworkProblemCounter}{1}
\nobreak\extramarks{Problem \arabic{homeworkProblemCounter}}{}\nobreak{}

%
% Homework Problem Environment
%
% This environment takes an optional argument. When given, it will adjust the
% problem counter. This is useful for when the problems given for your
% assignment aren't sequential. See the last 3 problems of this template for an
% example.
%
\newenvironment{homeworkProblem}[1][-1]{
    \ifnum#1>0
        \setcounter{homeworkProblemCounter}{#1}
    \fi
    \section{Problem \arabic{homeworkProblemCounter}}
    \setcounter{partCounter}{1}
    \enterProblemHeader{homeworkProblemCounter}
}{
    \exitProblemHeader{homeworkProblemCounter}
}

%
% Homework Details
%   - Title
%   - Due date
%   - Class
%   - Section/Time
%   - Instructor
%   - Author
%

\newcommand{\hmwkTitle}{Assignment \#1}
\newcommand{\hmwkDueDate}{January 27, 2017}
\newcommand{\hmwkClass}{CSCI 1300}
\newcommand{\hmwkClassTime}{Section 100}
\newcommand{\hmwkClassInstructor}{Professor David Knox}
\newcommand{\hmwkAuthorName}{\textbf{John Keller}}

%
% Title Page
%

\title{
    \vspace{2in}
    \textmd{\textbf{\hmwkClass:\ \hmwkTitle}}\\
    \normalsize\vspace{0.1in}\small{Due\ on\ \hmwkDueDate\ at 12:30pm}\\
    \vspace{0.1in}\large{\textit{\hmwkClassInstructor\ \hmwkClassTime}}
    \vspace{3in}
}

\author{\hmwkAuthorName}
\date{}

\renewcommand{\part}[1]{\textbf{\large Part \Alph{partCounter}}\stepcounter{partCounter}\\}

%
% Various Helper Commands
%

% Useful for algorithms
\newcommand{\alg}[1]{\textsc{\bfseries \footnotesize #1}}

% For derivatives
\newcommand{\deriv}[1]{\frac{\mathrm{d}}{\mathrm{d}x} (#1)}

% For partial derivatives
\newcommand{\pderiv}[2]{\frac{\partial}{\partial #1} (#2)}

% Integral dx
\newcommand{\dx}{\mathrm{d}x}

% Alias for the Solution section header
\newcommand{\solution}{\textbf{\large Solution}}

% Probability commands: Expectation, Variance, Covariance, Bias
\newcommand{\E}{\mathrm{E}}
\newcommand{\Var}{\mathrm{Var}}
\newcommand{\Cov}{\mathrm{Cov}}
\newcommand{\Bias}{\mathrm{Bias}}

\begin{document}

\maketitle

\pagebreak


\lstset{basicstyle=\fontsize{9}{10}\ttfamily,
  mathescape=true,
  escapeinside=||,
  autogobble,
  tabsize=4,
  breaklines=true,
  commentstyle=\color{gray},
  morecomment=[l]{//},
  postbreak=\raisebox{0ex}[0ex][0ex]{\ensuremath{\color{black}\hookrightarrow\space}}}


\begin{enumerate}
	\item Using those three rates of change, and a current U.S. population of 318,933,342, write a program to calculate the U.S. population in exactly one year (365 days). Your algorithm should output the result of your calculations.

\begin{lstlisting}
	// Test results of calculations are commented at end of each line
	current_population = 318933342;
	days_past = 365;
	immigrants = 24 * 60 * 24 * days_past; // 12614400
	deaths = 2 * 60 * 24 * days_past; // 1051200
	births = 8 * 60 * 24 * days_past; // 4204800
	final_population = immigrants + births + current_population - deaths; // 334701342
	output(final_population);
\end{lstlisting}
	
	\item A day has 86,400 seconds (24*60*60). Given a number of seconds in the range of 0 to 86,400, output the time as hours, minutes, and seconds for a 24­ hour clock. For example, 70,000 seconds is 19 hours, 26 minutes, and 40 seconds. Your program should have user input that is the number of seconds to convert, and then use that number in your calculations. Your output should be displayed as “The time is X hours, Y minutes, and Z seconds”.

\begin{lstlisting}
	seconds = input("Enter the number of seconds you want to convert!");
	if seconds less than 86400 // Error handling
		output("Error: You entered more seconds than are in a day!");
	else // No error; let's run the calculations
		// The Math.floor() command removes the decimals from each number
		hours = Math.floor(seconds / 86400 * 24); 
		minutes = Math.floor(((seconds / 86400 * 24) - hours) * 60);
		seconds = Math.floor((((seconds / 86400 * 24) - hours) * 60 - minutes) * 60);
		output("The time is "+hours+" hours, "+minutes+" minutes, and "+seconds+" seconds");
		// The plus signifies that the numbers and strings are combined for the human readable output
\end{lstlisting}

	\item In science, temperature is always described in Celsius, but in the U.S. we tend to use Fahrenheit temperatures. Write an algorithm to convert a Celsius temperature into Fahrenheit. (Subtracting thirty ­two from the Fahrenheit value and taking five ninths of that result will provide the Celsius value)
\begin{lstlisting}
	c = input("Enter the degree C you want to convert!");
	f = (c-32)*(5/9); // Subtracts 32, then multiplies it by 5/9.
	output(f);
\end{lstlisting}

	\item Write an algorithm that asks a user to enter a number between 1 and 10. (This range includes the numbers 1 and 10.) When they enter the number, check that it is actually between 1 and 10. If it is not, ask them to enter a number again. Continue to ask until they enter a valid number. Once their number is valid, output the number.
	
\begin{lstlisting}
	number = input("Enter a number between 1 and 10");
	while number greater than 10 or less than 0
		number = input("Error: Enter a number between 1 and 10"); // Let's try to get a number between 1 and 10 again
	output(number);
\end{lstlisting}

	\item Write an algorithm that asks a user the miles per gallon of their car. If they say greater than 30, output “Nice job”. If they say between 15 and 29, output “Not great, but okay.” If they say less than 15, output “So bad, so very, very bad.”

\begin{lstlisting}
	mpg = input("Enter the number of miles per gallon for your car");
	if mpg greater than or equal to 30
		output("Nice job");
	else if mpg greater than 15
		output("Not great, but okay");
	else 
		output("So bad, so very, very bad.");
\end{lstlisting}

	\item In text­-based choose your own adventure games, the game player is presented with choices throughout the game and then the game responds based on the user’s choice.
Write the algorithm for a simple choose your own adventure game where the user has three choices: \begin{itemize}
	\item Fight the dragon
	\item Go home
	\item Save the princess
\end{itemize}
The game should repeatedly ask the user which of the three options they want to do until the user says “Go home”. When “Go home” is selected, the loop should exit, which effectively ends the game.\newline\newline
If the user selects “Fight the dragon”, the algorithm should output “You win!”. If “Save the princess” is selected, the algorithm should output “You saved the princess”. If “Go home” is selected, the algorithm should output “Wimp”. \newline\newline
You can set up your algorithm to check for the user’s input in any way you like. Checking for the actual words, such as “Go home” is one option. If you want to assign a number to each option and check for the number, that also works.
	
\begin{lstlisting}
	// Note: The instructions for this problem were very confusing, I did the while to the best of my ability.
	game_choice = input("Enter the game you want to play: Fight the dragon, Go home, or Save the princess");
	while game_choice is not "Go home"
		game_choice = input("Enter the game you want to play: Fight the dragon, Go home, or Save the princess");
	output("The game is effectively over.");
	game_choice = input("Enter the game you want to play: Fight the dragon, Go home, or Save the princess");
	if game_choice is "Fight the dragon"
		output("You win!");
	else if game_choice is "Save the princess"
		output("You saved the princess");
	else if game_choice is "Go home"
		output("Wimp");
\end{lstlisting}


	\item Write an algorithm for playing a robotic version of the “treasure hunt” game. The “treasure hunt” game involves players receiving clues that guide them to other clues, until eventually the last clue leads them to a “treasure.” For example, at the beginning of the game the players might receive a slip of paper that says “go to the big oak tree,” at the oak tree they might find another slip of paper that says “try swimming,” so they go to the swimming pool to look for a third clue, and so forth.\newline
		
		Robots can’t read slips of paper, so the “clues” for a robot treasure hunt are represented by the colors of tiles. Furthermore, robots aren’t smart enough to think of (or find) oak trees or swimming pools, so every clue in a robot treasure hunt just requires a robot to move one tile in some direction. After doing so, the robot examines the color of the tile it is then standing on to figure out where to go next, and so forth.\newline
		
		Here are the specific rules for interpreting tile colors as clues:
		
		\begin{itemize}
			\item White tile means that the next clue is the tile directly in front of the robot
			\item Blue tile means that the next clue is the tile to the robot’s left
			\item Green tile means that the next clue is the tile to the robot’s right
			\item Black tile means that the robot should move two tiles backward, then
continue interpreting clues based on its new heading
			\item Yellow tile is a treasure — it marks the end of a part of a treasure hunt
		\end{itemize}
		
		\textbf{Rules for your algorithm:}
		\begin{itemize}
			\item Robot can only move forward, turn left, or turn right.
			\item The robot can detect the color of the position it is current on.
		\end{itemize}
		
\begin{lstlisting}
	current_color = random("White","Blue","Green","Black","Yellow"); // Start off the game by generating a random color
	
	output("Welcome to the \"Treasure Hunt\" game!");
	
	while current_color is not "Yellow"
		if current_color is "White"
			if input("The next clue is in front of you. Would you like to move one block forward? y/n") is "y"
				current_color = random("White","Blue","Green","Black","Yellow");
			else
				exit;
		if current_color is "Blue"
			if input("The next clue is on your left. Would you like to move there? y/n") is "y"
				current_color = random("White","Blue","Green","Black","Yellow");
			else
				exit;
		if current_color is "Green"
			if input("The next clue is on your right. Would you like to move there? y/n") is "y"
				current_color = random("White","Blue","Green","Black","Yellow");
			else
				exit;
		if current_color is "Black"
			if input("Uh oh, you need to move two tiles backwards. Would you like to move there? y/n") is "y"
				current_color = random("White","Blue","Green","Black","Yellow");
			else
				exit;
	if current_color is "Yellow"
		output("You've found the treasure! The game is now over.");
	 
\end{lstlisting}
	
\end{enumerate}



\end{document}
